\section{Przestrzenie liniowe}

\subsection{Konwencja sumacyjna}

W sumach postaci:
\[A=\sum\limits_{i=1}^{n}\alpha_i\beta_i=\alpha_1\beta_1+\alpha_2\beta_2+\cdots\alpha_n\beta_n,\]
\[B_i = \sum\limits_{j=1}^{n} \alpha_{ij}\beta_j = \alpha_{i1}\beta_1 + \alpha_{i2}\beta_2+\cdots+\alpha_{in}\beta_n,\]
\[C=\sum\limits_{i=1}^n \alpha_{ii} = \alpha_{11} + \cdots + \alpha_{nn}\]
pomijamy symbol $\sum$, jeżeli wskaźniki podlegające sumowaniu, zwane \textbf{martwymi}, powtarzają się.

Zatem piszemy:
\[A=\alpha_i\alpha_b,\]
\[B_i=\alpha_{ij}\beta_{j},\]
\[C=\alpha_{ii}\dend\]

Oznaczenie wskaźnika martwego nie jest istotne:
\[A=\alpha_i\beta_i=\alpha_k\beta_k=\alpha_1\beta_1+\cdots\alpha_n\beta_n \dend\]
Wskaźniki nie podlegające sumowaniu zwane wolnymi, muszą być jednakowe po obu stronach równości. Np.:
\[B_\mathbf{i} = \alpha_{\mathbf{i}j}\beta_j \dend\]

Wskaźniki umieszczamy także na \textbf{górnym poziome} (i bywa, że po lewej stronie \textbf{litery rdzeniowej}). Na przykład:
\[A=\alpha_i^i,\]
\[B^j = a^{ij}b_i,\]
\[\tensor[^k_r]{C}{}= \tensor[^r_k]{\alpha}{_i}\beta_i \dend\]

Nie należy wskaźników na górnym poziomie rozumieć jako wykładników potęg.
Wielkości potęgowane umieszczamy w nawiasach:
\[(\alpha^i)^2,\]
\[(\alpha^i)^j\beta_j,\]
\[(\alpha_{ij})^k\gamma_k \dend\]

Jeżeli wyróżniamy 'i-ty' składnik sumy
\[\alpha_i\beta_i = \alpha_1\beta_1+\cdots+\alpha_n\beta_n,\]
to wskaźnik tego składnika podkreślamy, czyli:
\[\alpha_i\beta_i = \alpha_1\beta_1+\cdots + \alpha_{\underline{i}} \beta_{\underline{i}}+ \cdots+\alpha_n\beta_n,\]

Wielkościami numerowanymi za pomocą wskaźników mogą być wielkości \textbf{liczbowe} i \textbf{wektorowe} (a także inne obiekty i struktury matematyczne).

\subsection{Pojęcie przestrzeni liniowej}

\begin{mydef}
  \textbf{Przestrzenią liniową} ( \textbf{wektorową} ) nad ciałem liczb rzeczywistych $\real$ nazywamy niepusty zbiór $\vec{\set{V}}$ elementów, zwanych \textbf{wektorami} (oznaczanych przez $\vec{x}$, $\vec{y}$, ...),
  wraz z dwoma działaniami:
  \begin{enumerate}
   \item \textbf{sumą wektorów} \[\vec{\set{V}} \times \vec{\set{V}} \to \vec{\set{V}}; \; \vec{z} = \vec{x} + \vec{y},\]
   \item \textbf{iloczynem wektora przez liczbę} \[\real \times \vec{\set{V}} \to \vec{\set{V}}; \; \vec{z} = \alpha \vec{x},\]
  \end{enumerate}
  spełniającymi warunki:
  \begin{enumerate}
   \item łączności dodawania \[(\vec{x} + \vec{y}) + \vec{z} = \vec{x} + (\vec{y} + \vec{z}),\]
   \item przemienności dodawania \[\vec{x} + \vec{y} = \vec{y} + \vec{x},\]
   \item rozdzielności dodawania względem mnożenia \[\alpha (\vec{x} + \vec{y}) = \alpha \vec{x} + \alpha \vec{y},\] \[(\alpha + \beta) \vec{x} = \alpha \vec{x} + \beta \vec{x},\]
   \item łączności iloczynu \[\alpha(\beta \vec{x}) = (\alpha \beta) \vec{x}\]
   \item istnienia wektora zerowego \[\vec{0} + \vec{x} = \vec{x} + \vec{0} = \vec{x}\]
   \item istnienia wektora przeciwnego $-\vec{x}$ do wektora $\vec{x}$ \[\vec{x} + (-\vec{x}) = (-\vec{x}) + \vec{x} = \vec{x} - \vec{x} = \vec{0}\]
   \item niezmienności wektora mnożonego przez 1 \[1\vec{x} = \vec{x}\]
  \end{enumerate}
  dla dowolnych wektorów $\vec{x}$, $\vec{y}$, $\vec{z}$ oraz liczb $\alpha$, $\beta$.
\end{mydef}

\begin{example}
   \underline{Przestrzeń arytmetyczna}
   
   Zbiór \[\real^n = \{\vec{x} \df (x_1,\cdots,x_n) \ozn (x_i); x_i \in \real \text{dla} i = 1,\cdots,n\},\]
   czyli \[\real^n = \underbrace{\real \times \cdots \times \real}_{n\; \text{razy}},\]
   z działaniami
   \[\vec{x} + \vec{y} \df (x_i+y_i) = (x_1+y_1, \cdots, x_n+y_n),\]
   \[\alpha \vec{x} \df (\alpha x_i) = (\alpha x_1, \cdots, \alpha x_n),\]
   dla dowolnych $\vec{x} = (x_i)$, $\vec{y} = (y_i) \in \real^n$ i $\alpha \in \real$, jest przestrzenią wektorową. Wektorem zerowym jest ciąg $n$ zer:
   \[\vec{0} = (0,\cdots,0)\]
   a wektorem przeciwnym do wektora - ciągu $\vec{x} = (x_i)$ jest ciąg liczb przeciwnych
   \[-\vec{x} = (-x_i) = (-x_1, \cdots, -x_n \dend\]
   
   Dla $n=1$ wnioskujemy, że $\real$ jest także przestrzenią wektorową przy \[\real^1 \ozn \real, \;\; (x_1) \ozn z \dend\]
\end{example}

\begin{example}
   \underline{Przestrzeń funkcyjna}
   
   Niech $\Omega$ dowolny zbiór, a $\vec{\set{V}}$ dowolna przestrzeń wektorowa. Zbiór funkcji (odwzorowań)
   \[F(\Omega, \vec{\set{V}}) = \{\vec{x}=\vec{f}(\xi), \xi\in\Omega\}\]
   wraz z działaniami:
   \[(\vec{f} + \vec{g})(\xi) \df \vec{f}(\xi) + \vec{f}(\xi), \xi \in \Omega,\]
   \[(\alpha \vec{f})(\xi) \df \alpha \vec{f}(\xi), \xi \in \Omega\]
   dla dowolnych $\vec{x} = \vec{f}(\xi)$, $\vec{g}=\vec{g}(\xi)$ ($\xi\in\Omega$) i $\alpha \in \real$ tworzy przestrzeń wektorową.
\end{example}

\begin{example}
  \underline{Przestrzeń ciągów}
  
  Niech $\{\vec{\set{V}}_i, i \in I \}$ ($I$ - skończony lub przeliczalny zbiór numerów/wskaźników) będzie rodziną przestrzeni liniowych. Zbiór \[\vec{\set{V}} = \{\vec{x} = (\vec{x}_i; \vec{x}_i \in \vec{\set{V}}_i \text{dla} i \in I\}\]
  z działaniami:
  \[\vec{x}+\vec{y}=(\vec{x}_i+\vec{y}_i),\]
  \[\alpha \vec{x} = (\alpha \vec{x}_i,\]
  dla dowolnych $\vec{x} = (\vec{x}_i)$,   $\vec{y} = (\vec{y}_i) \in \vec{\set{V}}$ oraz $\alpha \in \real$, jest przestrzenią wektorową.
  
  Na przykład, gdy $I = \nat$, to $\vec{x} = (\vec{x}_i) = (\vec{x}_1,\cdots,\vec{x}_n,\cdots)$ - ciąg wektorów a gdy ponadto $\vec{\set{V}}_i = \real$ dla wszystkich $i in I = \nat$, to $\vec{x} = (x_i) = (x_1, \cdots, x_n, \cdots)$ -- ciąg liczbowy.
\end{example}

\begin{mydef}
   Zbiór $\vset{U}$ zawarty w przestrzeni wektorowej (liniowej) $\vset{V}$ nazywamy \underline{podprzestrzenią} (liniową) przestrzeni $\vset{V}$, jeżeli $\vset{U}$ z działaniami określonymi w $\vset{V}$ i ograniczonymi do $\vset{U}$ stanowi przestrzeń wektorową.
\end{mydef}

\begin{tw}
   (Kryterium podprzestrzeni). Jeśli dla każdych $\vec{x}, \vec{y} \in \vset{U} \subset \vset{V}$ i $\alpha \in \real$ jest:
   \[\vec{x} + \vec{y} \in \vset{U},\]
   \[\alpha \vec{x} \in \vset{U},\]
   to $\vset{U}$ jest podprzestrzenią (liniową) przestrzeni $\vset{V}$.
\end{tw}

\begin{mydef}
   Niech $\{\vec{v}_1, \cdots, \vec{v}_m\} \subset \vset{V}$ dowolny skończony podzbiór wektorów przestrzeni $\vset{V}$ i niech $(\vec{v}_i)\ozn (\vec{v}_i, \cdots, \vec{v}_m)$.
   Wektor
   \[\vec{x} = \xi_i \vec{v}_i,\]
   dla $(\xi_i) = (\xi_i, \cdots, \xi_m) \in \real^m$ nosi nazwę \underline{kombinacji liniowej} ciągu $(\vec{v}_i)$.
\end{mydef}

\begin{example}
   Niech
   \[\vset{u} \ozn \lin (\vec{v}_i) \df \{\vec{x} = \xi_i \vec{v}_i; (\xi_i) \in \real^m\}\]
   gdzie $\vec{v}_i$ dowolny ustalony podzbiór (ciąg, układ) $m$ wektorów przestrzeni wektorowej $\vset{V}$. Zbiór $\vset{U}$ (wszystkich kombinacji liniowych wektorów $(\vec{v}_1, \cdots, \vec{v}_m)$) jest podprzestrzenią liniową przestrzeni $\vset{V}$ (na mocy kryterium) -- tzw. \underline{przestrzenią generowaną} przez układ wektorów $(\vec{v}_1, \cdots, \vec{v}_m)$.
\end{example}

\begin{mydef}
   Niech $\vset{U}'$ i $\vset{U}''$ podprzestrzenie liniowe $\vset{V}$. Zbiór
   \[\vset{U} = \vset{U}' + \vset{U}'' = \{\vec{u} = \vec{u}' + \vec{u}''; \vec{u}' \in \vset{U}'; \vec{u}'' \in \vset{U}'' \},\]
   nosi nazwę sumy podprzestrzeni $\vset{U}'$ i $\vset{U}''$.
   Jest to podprzestrzeń $\vset{V}$. Jeśli ponadto $\vset{U}' \cap \vset{U}'' = \{\vec{0}\}$ to $\vset{U} \ozn \vset{U}' \oplus \vset{U}"$ jest tzw. \underline{sumą prostą} $\vset{U}'$ i $\vset{U}''$.
\end{mydef}

\subsection{Przestrzenie skończenie wymiarowe. Baza algebraiczna}

\begin{mydef} 
   Podzbiór ${\vec{e}_1, \cdots, \vec{e}_n} \ozn (\vec{e}_i) = (\vec{e}_i, \cdots, \vec{e}_n)$ przestrzeni wektorowej $\vset{V}$ nazywa się \underline{liniowo niezależnym}, jeżeli prawdziwa jest implikacja:
   \[\alpha_i \vec{e}_i = \vec{0} \to \alpha_i = 0\;\; \forall i = 1,\cdots, n \dend\]
\end{mydef}

\begin{mydef}
   Zbiór $\vset{B} = (\vec{e}_i)$ wektorów z przestrzeni $\vset{V}$ nazywamy \underline{bazą (algebraiczną)}, jeżeli:
   \begin{enumerate}
    \item $\vset{B}$ jest liniowo niezależnym
	\item $\lin \vset{B} = \vset{V}$ ($\vset{B}$ generuje $\vset{V}$).
   \end{enumerate}
\end{mydef}

\begin{mydef}
   Jeżeli w przestrzeni $\vset{V}$ istnieje baza n-elementowa $\vset{B}$, to $\vset{V}$ nazywamy \underline{n-wymiarową} (\underline{skończenie wymiarową} o wymiarze $n$) i piszemy
   \[\dim \vset{V} = n\]
\end{mydef}

\begin{info}
   Jeżeli istnieje w $\vset{V}$ baza n-elementowa, to istnieje nieskończenie wiele baz i każda jest n-elementowa.
\end{info}

\begin{info}
   Jeżeli $\vset{B}$ jest n-elementową bazą przestrzeni $\vset{V}$, to
   \[\forall \vec{x} \in \vset{V} \exists (x^i) \in \real^n \;\; \vec{x} = x^i \vec{e}_i,\]
   gdzie $(x^i)$ są tzw. \underline{współczynnikami rozkładu} lub \underline{współrzędnymi} wektorze $\vec{x}$ w bazie $\vset{B}$. Przy tym rozkład ten jest \underline{jednoznaczny}, bowiem
   \[\vec{x} = x^i \vec{e}_i = y^i \vec{e}_i \to (x^i - y^i) \vec{e}_i = \vec{0} \to x^i - y^i = 0 \;\; \forall i = 1,2,\cdots, n \dend\]
\end{info}

\begin{example}
    Bazą przestrzeni arytmetycznej $\real^n$ -- tzw. \underline{bazą standardową} - jest układ ciągów
    \[\vec{e}_1 = (1, 0, 0, \cdots, 0),\]
    \[\vec{e}_2 = (0, 1, 0, \cdots, 0),\]
    \[\vdots\]
    \[\vec{e}_n = (0, 0, 0, \cdots, 1)\]
    czyli
    \[\vec{e}_i = (\delta_{ij}) dla i=1,\cdots, n,\]
    gdzie
    \[\delta_{ij} \df CASES\]
    jest tzw. \underline{symbolem Kroneckera}.
\end{example}

\begin{info}
    Przestrzeń funkcyjna $F(\Omega, \vset{V})$ - nieskończenie wymiarowa.
\end{info}

\begin{info}
    Jeżeli $(\vec{e}_1, \cdots, \vec{e}_m) \subset \vset{V}$ liniowo niezależny, to $\lin(\vec{e}_1, \cdots, \vec{e}_k) \oplus \lin(\vec{e}_{k+1}, \cdots, \vec{e}_m) = \lin(\vec{e}_1, \cdots, \vec{e}_m)$ oraz $\dim \lin (\vec{e}_1, \cdots, \vec{e}_m) = m$.
\end{info}


\begin{info}
    Niech $(\vec{e}_i$ i $(\vec{e'}_{i}) \ozn (\vec{e}_{i'}) (i,i'=1,\cdots,n)$ dwie bazy przestrzeni n-wymiarowej $\vset{V}$. Zatem
    \[\vec{e}_{i'} \ A_{i'}^i \vec{e}_i, \;\; \vec{e}_{i} \ A_{i}^{i'} \vec{e}_{i'},\]
    a w konsekwencji
    \[\vec{e}_{i'} = A_{i'}^i \vec{e}_i = A_{i'}^i A_{i}^{j'} \vec{e}_{j'},\]
    skąd
    \[A_{i'}^i A_{i}^{j'} = \delta^{i'}_{j'} = CASES,\]
    czyli
    \[[A^{i}_{i'}][A^{j'}_{i}]=[\delta_{i'}^{j'}],\]
    lub
    \[\mat{A}\mat{A}' = \mat{I}\;\; (\mat{A}' = \mat{A}^{-1}),\]
    w notacji macierzowej, przy
    \[\mat{A} = [A^i_{i'}], \;\;  \mat{A}' = [A^{i'}_{i}],\]
    tzw \underline{macierzach transformacji baz} (z $\vec{e}_i$ do $\vec{e}_{i'}$ i na odwrót).
\end{info}

Jeżeli $\det \mat{A} > 0$, to bazy $(\vec{e}_{i})$ i $(\vec{e}_{i'})$ są \underline{zgodnie zorientowane}.

Niech 
\[\vec{x} = x^i \vec{e}_i = x^{i'} \vec{e}_{i'} \dend\]
Wobec
\[x^{i'} \vec{e}_{i'} = x^{i'} A_{i'}^i \vec{e}_i = x^i \vec{e}_i = x^i A^{i'}_i\vec{e}_{i'},\]
mamy
\[x^{i'} = A_i^{i'} x^i, \;\; x^{i} = A_{i'}^{i} x^{i'},\]
czyli w notacji macierzowej
\[[x^{i'}] = [A_i^{i'}][x^i], \;\; [x^i]=[x_{i'}^i][x^{i'}]\]

\begin{info}
   Niech $\vset{B} = (\vec{e}_i)$ ustalona baza przestrzeni $\vset{V}$.
   Odwzorowanie
   \[\vset{i}_{\vset{B}} : \vset{V} \to \real^n; \; \vset{i}_{\vset{B}}(\vec{x}) = (x^i), \; \vec{x} = x^i \vec{e}_i,\]
   ustala tzw. \underline{izomorfizm} $\vset{V}$ i $\real^n$ w danej bazie $\vset{B}$.
\end{info}


\subsection{Przestrzenie unormowane}

\begin{mydef}
   Przestrzenią (liniową, wektorową) \underline{unormowaną} nazywamy parę (układ) $\{\vset{V}, |\cdot|\}$, gdzie $\vset{V}$ jest przestrzenią wektorową, a $|\cdot|$ -- odwzorowaniem, zwanym \underline{normą}, o następujących własnościach:
   \[|\cdot|: \vset{V} \to <p, \infty) \subset \real,\]
   \begin{enumerate}
    \item $|\vec{x}| = 0 \wtw \vec{x} = \vec{0}$,
    \item $|\alpha \vec{x}| = |\alpha|\,|\vec{x}| \;\; \forall \alpha \in \real, \forall \vec{x} \in \vset{V}$,
    \item $|\vec{x} + \vec{y}| \leq |\vec{x}| + |\vec{y}| \;\; \forall \vec{x}, \vec{y} \in \vset{V}$.
   \end{enumerate}
   Liczba $|\vec{x}|$ -- \underline{norma} lub \underline{długość} wektora $\vec{x}$.
\end{mydef}

\begin{info}
   \underline{Wektor jednostkowy} (inaczej wersor) -- wektor o długości jednostkowej ($\vec{i} \text{ - wersor} \wtw |\vec{i}| = 1$).
\end{info}

\begin{example}
   Przestrzeń wektorowa arytmetyczna $\real^n$ jest unormowana -- z normą:
   \[|\vec{x}| = \df \sum\limits_{i=1}^{n} |x_i|, \;\; \vec{x} = (x_i)\dend\]
\end{example}

\begin{example}
   Niech $\vset{V}$ przestrzeń unormowana z normą $|\cdot|$. Zbiór
   \[L^\infty(\Omega, \vset{V}) = \{\vec{f} \in F(\Omega, \vset{V}); \sup \limits_{\xi \in \Omega} |\vec{f}(\xi)| < \infty\},\]
   jest podprzestrzenią liniową przestrzeni $F(\Omega, \vset{V})$, co wynika z kryterium podprzestrzeni, a więc jest przestrzenią wektorową - tzw. \underline{przestrzenią funkcji ograniczonych}.
   Ponadto przestrzeń ta jest unormowana, gdyż 
   \[|| \vec{f} ||_\infty \df \sup \limits_{\xi\in\Omega} |\vec{f}(\xi)|,\]
   spełnia warunki definicyjne normy.
\end{example}

\begin{info} 
   Jeżeli $\vset{V}$ jest skończenie wymiarowa, to każde dwie normy na $\vset{V}$ są \underline{równoważne}, tzn. 
   \[\exists \alpha, \beta > 0 \;\; \alpha |\vec{x}|_1 \leq |\vec{x}|_2 \leq |\vec{x}|_1 \;\; \forall \vec{x} \in \vset{V}\dend\]
\end{info}

\begin{info}
   Jeżeli $\vset{U}$ jest podprzestrzenią liniową przestrzeni unormowanej $\vset{V}$ z normą $|\cdot|$, to $\vset{U}$ jest również \underline{unormowana} -- z normą $|\cdot|$ obciętą do $\vset{U}$.
\end{info}



\begin{tw}
   Przestrzeń wektorowa unormowana $\{ \vset{V}, |\cdot|\}$ jest ``automatycznie'' metryczna -- z metryką \underline{generowaną} przez normę:
   \[d(\vec{x}, \vec{y}) \df |\vec{y} - \vec{x}|, \;\; \forall \vec{x}, \vec{y} \in \vset{V},\]
   (jeśli wektory przestrzeni $\vset{V}$ potraktować także jako punkty przestrzeni $D = \vset{V}$).
\end{tw} 

\begin{mydef}
   Jeżeli przestrzeń unormowana $\vset{V}$ jest jako przestrzeń metryczna (z metryką generowaną przez normę) zupełna, to nazywa się \underline{przestrzenią Banacha}.
\end{mydef}

\begin{info}
   Każda skończenie wymiarowa i unormowana przestrzeń wektorowa jest przestrzenią Banacha (w tym $\real^n$).
\end{info}

\begin{info}
   Każda skończenie wymiarowa podprzestrzeń przestrzeni unormowanej jest domknięta.
\end{info}

\begin{info}
   Żadna nieskończenie wymiarowa przestrzeń Banacha $\vset{V}$ nie daje się przedstawić w postaci sumy $\vset{V}_1 \cup \vset{V}_2 \cup \cdots$ skończenie wymiarowych podprzestrzeni $\vset{V}$.
\end{info}

\begin{example}
   Niech $K$ podzbiór zwarty przestrzeni $\real^n$ i niech $C(K, \real^m)$ przestrzeń funkcji ciągłych (jako podprzestrzeń przestrzeni $F(K; \real^m)$). Przestrzeń ta jest przestrzenią Banacha z normą:
   \[||f|| = \sup\limits_{x \in K} |f(x)|,\]
   gdzie $|f(x)|$ -- dowolna norma w $\real^n$.
\end{example}

\begin{example}
   Niech $U$ oznacza podzbiór otwarty i mierzalny (w sensie Lebesgue'a) w $\real^n$ i niech $L(U; \real)$ oznacza zbiór wszystkich funkcji całkowalnych (w sensie Lebesgue'a) na $U$  o wartościach w $\real$. $L(U,\real)$ jest podprzestrzenią liniową przestrzeni funkcyjnej $F(U, \real)$, a ponadto przestrzenią Banacha z normą
   \[|f| = \int\limits_U |f(x)| dV,\]
   gdzie $dV$ -- miara w $U \subset \real^n$, $d V = dx_1 \cdot 
   ... \cdot dx_n$ przy $x = (x_1, ..., x_n)$.
\end{example}

\begin{mydef}
   Niech $(\vset{V}_n) = \{\vset{V}_n \subset \vset{V}; n \in \nat\}$ \underline{ciąg} podprzestrzeni liniowych przestrzeni unormowanej (Banacha) $\subset{V}$.
   Mówimy, że ciąg $(\vset{V}_n)$ \underline{aproksymuje} przestrzeń $\vset{V}$, jeżeli
   \[\forall \varepsilon > 0 \; \forall \vec{x} \in \vset{V} \; \exists N \in \nat \; \exists \vec{x}_N \in \vset{V}_N \; |\vec{x}-\vec{x}_N| < \varepsilon\]
   oraz, że ciąg $(\vset{V}_n)$ aproksymuje przestrzeń $\vset{V}$ \underline{jednostajnie}, jeżeli

   \[\forall \varepsilon > 0 \; \exists N \in \nat \; \forall \vec{x} \in \vset{V} \; \exists \vec{x}_N \in \vset{V}_N \; |\vec{x} - \vec{x}_N| < \varepsilon\].
\end{mydef}


\subsection{Przestrzenie unitarne}

\begin{mydef}
   \underline{Przestrzenią} (liniową) \underline{unitarną} (\underline{przestrzenią z iloczynem skalarnym}) nazywamy parę $\{\vset{V}; <\cdot, \cdot>\}$, gdzie $\vset{V}$ jest przestrzenią wektorową (liniową), a $<\cdot, \cdot>$ odwzorowaniem
   
   \[<\cdot, \cdot>: \vset{V} \times \vset{V} \to \real \; (<\vec{x}, \vec{y}> \ozn \vec{x} \cdot \vec{y}),\]
   zwanym \underline{iloczynem skalarnym} (produktem skalarnym), spełniającym warunki:
   
   \begin{enumerate}
    \item $\vec{x} \cdot \vec{x} \geq 0, \; \vec{x} \cdot \vec{x} = 0 \wtw \vec{x} = \vec{0}$,
    \item $\vec{x} \cdot \vec{y} = \vec{y} \cdot \vec{x}$,
    \item $(\alpha \vec{x}' + \beta \vec{x}'') \cdot \vec{y} = \alpha \vec{x}' \cdot \vec{y} + \beta \vec{x}'' \cdot \vec{y}$.
   \end{enumerate}

\end{mydef}

\begin{info}
   Przestrzeń unitarna $\{\vset{V}, <\cdot, \cdot>\}$ jest unormowana (\underline{automatycznie}) -- z normą:
   \[|\vec{x}| \df \sqrt{\vec{x} \vec{x}}, \; \vec{x} \in \vset{V},\]
   (tzw, \underline{normą generowaną przez iloczyn skalarny} a w konsekwencji przestrzenią metryczną -- z metryką generowaną przez normę:
   \[d(\vec{x}, \vec{y}) = |\vec{y}-\vec{x}| = \sqrt{(\vec{y}-\vec{x})\cdot()\vec{y}-\vec{x}}, \; \vec{x},\vec{y} \in \vset{V},\]
   (jeśli wektory $\vec{x}$, $\vec{y}$ potraktować jako punkty).
\end{info}

\begin{mydef}
   Wektory $\vec{x}$ i $\vec{y}$ nazywamy \underline{prostopadłymi} (lub \underline{ortogonalnymi} -- $\vec{x} \perp \vec{y}$), gdy $\vec{x} \cdot \vec{y} = 0$. Natomiast prostopadłe są zbiory $\vset{U}'$ i $\vset{U}''$ przestrzeni $\vset{V}$ ($\vset{U}' \perp \vset{U}''$), jeśli
   \[\forall \vec{x} \in \vset{U}' \; \forall \vec{y} \in \vset{U}''\; \vec{y} \perp \vec{x} \dend\]
\end{mydef}

\begin{info}
   Prawdziwa jest \underline{nierówność Cauchy'ego}
   \[|\vec{x} \cdot \vec{y} \leq |\vec{x}| |\vec{y}| \dend\]
\end{info}

\begin{example}
   w przestrzeni $\real^n$
   \[\vec{x} \cdot \vec{y} = x_i y_i, \;\; \vec{x} = (x_i), \; \vec{y} = (y_i),\]
   określa iloczyn skalarny, zwany \underline{standardowym}, który generuje w $\real^n$ tzw, \underline{standardowe} (lub \underline{euklidesowe}) normę i metrykę:
   \[|\vec{x}| = \sqrt{x_i x_i},\]
   \[d(\vec{x}, \vec{y}) = \sqrt{(y_i-x_i)(y_i-x_i)} \dend\]
\end{example}

\begin{example}
   Niech $U$ oznacza podzbiór otwarty i mierzalny (w sensie Lebesque'a) w $\real^n$ i niech 
   \[L^2_\rho (U, \real) = \{f \in F(U, \real); \int \limits_{U} \rho(x) f^2(x) dV < \infty\},\]
   oznacza zbiór funkcji całkowalnych z kwadratem z wagą $\rho$ na $U$ (w sensie Lebesgue'a), gdzie $\rho: U \to \real$ jet funkcją \underline{mierzalną nieujemną}, nazwaną \underline{wagą}. Zbiór ten jest podprzestrzenią liniową przestrzeni $F(U, \real)$, a więc jest przestrzenią wektorową, a przy tym \underline{unitarną} z iloczynem skalarnym:
   \[<f,g>_\rho = \int \limits_{U} \rho(x)f(x)g(x) dV,\]
   \[dV = dx_1\cdot ... \cdot dx_n \;\;\text{przy}\;\; x=(x1, ..., x_n) \dend\]
   Iloczyn skalarny generuje tzw. normę \underline{średniokwadratowa} (i odpowiednią metrykę średniokwadratową) -- z wagą $\rho$ lub ``bez wagi'':
   \[||g||_\rho = (<f,f>_\rho)^{1/2} = \left(\int\limits_U \rho(x) f^2(x) dV \right)^{1/2}\]
   \[||g|| = (<f,f>)^{1/2} = \left(\int\limits_U  f^2(x) dV \right)^{1/2}\]
\end{example}


\begin{example}
   Niech $\{\vset{V}_i; <\cdot, \cdot>_i\}$, $i \in \set{I}$ ($\set{I}$ -- skończony lub przeliczalny zbiór wskaźników) będzie rodziną przestrzeni unitarnych. Niech $\rho = (\rho_i)$ -- ciąg liczbowy ($\rho_i \in \real, i \in \set{I}$). Zbiór
   \[l^2_\rho = \left\{\vec{x} = (\vec{x}_i) \in \vset{V}; \; \vec{x}_i \in \vset{V}_i\; \forall i \in \set{i}, \; \sum\limits_{i \in\ \set{I}} \rho_i <\vec{x}_i, \vec{x}_i>_i < \infty \right\}\]
   jest podprzestrzenią liniową przestrzeni wektorowej
   \[\vset{V} = \{\vec{x} = (\vec{x}_i); \; \vec{x}_i \in \vset{V}_i, \; i in \set{I}\},\]
   a więc jest przestrzenią wektorową -- przy tym unitarną z iloczynem skalarnym:
   \[<\vec{x}, \vec{y}>_\rho = \sum\limits_{i\in\set{i}} \rho_i <\vec{x}_i, \vec{y}_i>_i\]
   i normą generowaną przez ten iloczyn:
   \[|\vec{x}|_\rho = \left( \sum\limits_{i\in\set{i}} \rho_i <\vec{x}_i, \vec{x}_i>_i \right)^{1/2} \dend\]
   W szczególności, gdy \underline{ciąg wagowy} jest \underline{jednostkowy}, 
   tzn. $\vec{\rho} = (1,1, ...)$, mamy przestrzeń z oznaczeniami:
   \[l^2(\vset{V}), \; <\cdot, \cdot>\; \text{i}\; |\cdot| \dend\]
   Jeżeli $\vset{V} = \real$ dla wszystkich $i\in\set{i}$, to mamy tzw. przestrzeń ciągów liczbowych (skończonych lub przeliczalnych) \underline{sumowalnych z kwadratem} -- z wagą $\rho$: (t.j. ciągiem wagowym $\rho = (\rho_i)$) lub ``bez wagi'' (tj. ciągiem wagowym jednostkowym):

   \[l^2_\rho \df \left\{ \vec{x} = (x_i); x_i \in\real\; \forall i \in\set{I}, \; \sum\limits_{i\in\set{I}} \rho_i x_i^2 < \infty \right\},\]
   \[ <\vec{x}, \vec{y}>_\rho = \sum\limits_{i\in\set{I}} \rho_i x_i y_i, \; |\vec{x}|_\rho = \sum\limits_{i\in\set{I}} \rho_i x_i^2,\]
   
   (z oznaczeniami $l^2$, $<\cdot, \cdot>$ i $|\cdot|$ przy ``braku wagi'').
   W przypadku, gdy $\set{I} = \{1,2,3,..,n\}$ przestrzeń $l^2$ pokrywa się z $\real^n$ ze standardowym iloczynem skalarnym i normą.
\end{example}

Następujące twierdzenia są poprawne w odniesieniu do przestrzeni unitarnej:
\begin{tw}
   (Pitagorasa) Jeżeli wektory $\vec{x}$ i $\vec{y}$ są prostopadłe ($\vec{x} \perp \vec{y}$), to 
   \[|\vec{x}+\vec{y}|^2 = |\vec{x}|^2 + |\vec{y}|^2 \dend\]
\end{tw}

\begin{tw} 
   Niech $\vset{U}$ podzbiór przestrzeni unitarnej $\vset{V}$. Zbiór $\vset{U}^{\perp} = \{\vec{y} \in \vset{V}; \; \{\vec{y}\} \perp \vset{U}\}$, zwany \underline{dopełnieniem ortogonalnym $\vset{U}$}, jest podprzestrzenią liniową $\vset{V}$.
   
   Jeśli ponadto $\vset{U}$, jest skończenie wymiarową podprzestrzenią $\vset{V}$, to $\vset{V} = \vset{U} \oplus \vset{U}^\perp$.
\end{tw}

\begin{tw}
   (o rzucie ortogonalnym) Niech $\vset{V}$ przestrzeń unitarna, a $\vset{U}$ jej skończenie wymiarowa podprzestrzeń liniowa, Wtedy
   \[\forall \vec{x} \in \vset{V} \; \exists \vec{a} \in \vset{U} \vec{x} = \vec{a} + \vec{y}, \; \{\vec{y}\} \perp \vset{U} \dend\]
   
   Wektor $\vec{a} \ozn \rzut_{\vset{U}}\vec{x}$ nosi nazwę \underline{rzutu ortogonalnego} $\vec{x}$ na $\vset{U}$.
\end{tw}

\begin{tw}
   Zbiór $(\vec{e}_1, ..., \vec{e}_m)$ jest liniowo niezależny w $\{\vset{V}; <\cdot, \cdot>\}$ wtw. $G = det[\vec{e}_i \cdot \vec{e}_j] \neq 0$ ($G$ -- tzw wyznacznik Grama).
\end{tw}

\begin{mydef} 
   Przestrzeń unitarna $\vset{V}$ nazywa się \underline{przestrzenią Hilberta}, jeżeli $\vset{V}$ jako przestrzeń metryczna (z metryką generowaną przez normę generowaną przez iloczyn skalarny) jest \underline{zupełna}.
\end{mydef}

\begin{info}
   Przestrzeń Hilberta jest Banacha z norma generowaną przez iloczyn skalarny, a przestrzeń unitarna będąca przestrzenią Banacha (jako przestrzeń unormowana jest przestrzenią Hilberta.
\end{info}

\begin{tw}
   (o rzucie ortogonalnym - c.d.) Niech $\vset{V}$ przestrzeń Hilberta, a 
   $\vset{Z}$ jej podzbiór wypukły i domknięty. Wtedy
   \[\forall \vec{x} \in \vset{V} \; \exists \vec{a} \in \vset{Z} \; |\vec{x} - \vec{a}| = \inf |\vec{x} - \vec{z}| \; \vec{z} \in \vset{Z},\]
   a przy czym $\vec{a} \ozn \rzut_{\vset{Z}} \vec{x}$ jest określony jednoznacznie (stosujemy również oznaczenie:
   \[ P_{\vset{Z}}: \vset{V} \to \vset{Z}, P_{\vset{Z}} \vec{x} = \rzut_{\vset{Z}} \vec{x} ).\]
\end{tw}

\begin{tw}
   Podprzestrzeń liniowa $\vset{U}$ przestrzeni Hilberta $\vset{V}$ jest gęsta wtw. gdy jedynym elementem $\vec{x} \in \vset{V}$ ortogonalnym do $\vset{U}$ jest $\vec{x} = \vec{0}$.
\end{tw}

\begin{mydef}
   Niech $(\vec{V}_n) = \{\vec{V}_n \subset \vec{V}; n \in \nat\}$ ciąg liniowych domkniętych podprzestrzeni Hilberta $\vec{V}$. Mówimy, że $\vec{V}$ jest \underline{sumą hilbertowską} podprzestrzeni $\vec{V}_n$, co oznaczamy przez $\vec{V} = \oplus_{n\in\nat} \vset{V}$, jeżeli
   \begin{enumerate}
    \item $\vset{V}_n$ są wzajemnie ortogonalne, tzn $\forall n\neq m \vset{V}_n \perp \vset{V}_m$
    \item podprzestrzeń liniowa rozpięta przez $\vset{V}_n$ jest gęsta w $\vset{V}$, tzn.
   \[\clos \left(\lin \bigcup\limits_{n \in \nat} \vset{V}_n\right) = \vset{V},
   \]
    czyli
    \[\forall \vec{x} in \vset{V}\; \forall n \in \nat \; \exists \vec{x}_n \in \vset{V}_n \lim\limits_{N\to\infty} \sum\limits_{n=1}^N \vec{x}_n = \vec{x},\]
    piszemy także
    \[\lim\limits_{N\to\infty} \sum\limits_{n=1}^N \vec{x}_n \ozn \sum\limits_{n=1}^{\infty} \vec{x}_n \ozn \sum\limits_{n\in\nat}\vec{x}_n\]
    i nazywamy tę granicę \underline{sumą nieskończoną} albo \underline{szeregiem} elementów $\vec{x}_n$ (ciągu $(\vec{x}_n)$ elementów z $\vset{V}_n$).
   \end{enumerate}
\end{mydef}

\begin{info}
   W powyższej definicji ciąg przestrzeni może być w szczególności ``skończony'', tzn $\exists n^* \in \nat \; n > n^* \; \vset{V}_n = \{\vec{0}\}$, a $\sum\limits_{n=1}^{\infty} \vec{x} = \sum\limits_{n=1}^{n^*} \vec{x}_n$.
\end{info}

\begin{tw}
   Niech $\vset{V}$ będzie sumą hilbertowską podprzestrzeni $\vec{x}_n$. Wtedy, dla dowolnego $x \in \vset{V}$
   \begin{enumerate}
    \item $\vec{x} = \sum\limits_{n=1}^{\infty} \vec{x}_n$,
    \item $|\vec{x}|^2 = \sum\limits_{n=1}^{\infty} |\vec{x}_n|^2$,
   \end{enumerate}
   gdzie 
   \[\vec{x}_n = \rzut_{\vset{V}_n} \vec{x} = P_{\vset{V}_n} \vec{x} \;\; \forall n \in \nat \dend\]

   Również na odwrót, dla każdego ciągu $(\vec{x}_n) = \{\vec{x}_n \in \vset{V}_n\}; \; n \in \nat$ takiego, że $\sum\limits_{n=1}^{\infty} |\vec{x}|^2 < \infty$, suma (szereg)
   \[\vec{x} = \sum\limits_{n=1}^{\infty} \vec{x}_n,\]
   istnieje (szereg jest zbieżny) i $\vec{x}_n = \rzut_{\vset{V}_n} \vec{x}\;\; \forall n \in \nat$ oraz prawdziwa jest równość 2).
\end{tw}
